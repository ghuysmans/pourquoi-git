\documentclass[compress]{beamer}
\usepackage{graphicx}
\usepackage{enumerate}
\usepackage[utf8]{inputenc}
\usepackage[T1]{fontenc}
\usepackage[francais]{babel}
\setbeamertemplate{caption}[numbered]
\title{Introduction à Git}
\author{Guillaume \textsc{Huysmans}}
\bibliographystyle{plain}
\newenvironment{envie}
	{\begin{block}{Ce qu'on a envie de dire...}}
	{\end{block}}
\begin{document}
\begin{frame}
	\maketitle
\end{frame}
\begin{frame}
	\frametitle{Plan}
	\tableofcontents
\end{frame}


\section{Problématique}
\begin{frame}
	\frametitle{Collaboration}
	Un jour ou l'autre, on aura un projet à faire à plusieurs.
	Il devient nécessaire de fusionner les changements et de se synchroniser.

	\begin{envie}
	C'est faisable manuellement ! \pause
	\end{envie}

	\begin{itemize}
		\item par e-mail / réseau social \pause (trop volumineux ?) \pause
		\item par clé USB \pause (copier quoi ? dans quel sens ?) \pause
		\item à la Google Docs \pause (et sans connexion ?) \pause
		\item par un Drive quelconque \pause (conflits ?)
	\end{itemize}
\end{frame}

\begin{frame}
	\frametitle{Un bon contrôle de versions}
	\begin{itemize}
		\item Synchronisation (semi-)automatique
		\item Travail hors-ligne
		\item Décentralisé : pas d'obligation à utiliser un unique serveur
		\item Rapide (et cross-platform : GNU/Linux, Windows, OS X)
		\item Historisé :
			\begin{itemize}
				\item Liste lisible des derniers changements
				\item Qui a fait quoi ?
				\item Possibilité de revenir en arrière
			\end{itemize}
	\end{itemize}
\end{frame}

\section{Solution}
\begin{frame}
	\frametitle{Git}
	Selon les critères exposés juste avant, Git,
	développé par Linus Torvalds (créateur de Linux),
	est un bon système de contrôle de versions. %(VCS en angais).

	\begin{itemize}
		\item Utilisations : \pause
			\begin{itemize}
				\item Le noyau Linux :
					\begin{itemize}
						\item Plus de 18 millions de lignes
						\item 91 Mo de texte compressé ! \pause
					\end{itemize}
				\item Des cours et manuels collaboratifs (à la Wiki) \pause
				\item \url{https://github.com/steeve/france.code-civil} \pause
			\end{itemize}
		\item Hébergement connu : \url{https://github.com} \pause
		\item Interfaces graphiques \pause
		\item Gère avant tout du texte (gênant ?)
	\end{itemize}
\end{frame}

\begin{frame}
	\frametitle{Du texte ?!}
	Le code source d'un programme est \emph{déjà} du texte.
	\pause

	Les documents peuvent aussi en être : il suffit de les rédiger en
	\LaTeX{} (comme cette présentation), Markdown... !
	\pause

	\begin{envie}
		Qu'est-ce qu'on y gagne ? \pause
	\end{envie}

	Avantages :
	\begin{itemize}
		\item Fusion parfois automatique des changements \pause
		\item Reproductibilité : le résultat sera le même partout. \pause
			\begin{itemize}
				\item La fusion n'est faite qu'une fois. \pause
				\item \LaTeX{} n'a pas autant de fantaisie que Word.
			\end{itemize}
	\end{itemize}
\end{frame}

%*Git permet de maintenir facilement un historique de changements.*
%Comment ? En calculant les changements par rapport à une version connue.
%À chaque ensemble de modifications on associe un message qui les... :
%- résume dans le titre
%- explique dans le corps du message


\end{document}
